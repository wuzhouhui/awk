% vim: ts=4 sts=4 sw=4 et tw=75

\chapter{Awk 语言}
\label{chap:the_awk_language}

\marginpar{21}
这一章解释---大部分都带有例子---构成一个 awk 程序的构造要素. 由于这次要描述
的是整个语言, 所以材料会非常琐细, 所以我们建议读者只需要浏览一下即可, 当
有必要的时候再回来查看细节.

最简单的 awk 程序是一个由多个 \patact 构成的序列:
\begin{pattern}
\textit{pattern} \texttt{\{} \textit{action} \texttt{\}} \par
\textit{pattern} \texttt{\{} \textit{action} \texttt{\}} \par
...
\end{pattern}
在某些语句中, 模式可以不存在; 还有些语句, 动作及其包围它的花括号也可以不存
在. 如果你的程序经过 awk 检查后没有发现语法错误, 它就会每次读取一个输入行,
对读取到的每一行, 按顺序检查每一个模式. 对每一个与当前行匹配的模式, 对应
的动作就会被执行. 一个缺失的模式匹配每一个输入行, 因此每一个不带有模式的动
作对每一个输入行都会执行. 只含有模式而没有动作的语句, 会打印每一个匹配模式
的输入行. 大部分情况下, 在这一章出现的术语 ``输入行'' 与 ``记录'' 被当作
一对同义词. 在 \ref{sec:input} 节, 我们都会讨论多行记录, 多行记录指的是一
个记录包含了多个行.

这一章的第一节会详细描述模式. 第二节通过描述表达式, 赋值语句与流程控制语句,
展开对动作的讨论. 剩下的小节包括函数定义, 输出, 输入, 以及 awk 如果调用其
他程序. 大部分小节都包含对主要性质的总结.

\subsection{输入文件\filename{countries}}
\label{subsec:the_input_file_countries}

作为本章许多 awk 程序的输入数据, 我们将使用文件 \filename{countries}. 每一
行都包括一个国家的名字, 面积 (以千平方英里为单位), 人口 (以百万为单位), 以
及这个国家所在的大陆. 数据来源于 1984 年, 苏联被归到了亚洲. 在文件里, 四列
数据用制表符分隔; 用一个空白符分隔 \texttt{North} (\texttt{South}) 与
\texttt{America}.

文件 \filename{countries} 包含下面几行:
\marginpar{22}
\begin{file}
    USSR        8649    275     Asia
    Canada      3852    25      North America
    China       3705    1032    Asia
    USA         3615    237     North America
    Brazil      3286    134     South America
    India       1267    746     Asia
    Mexico      762     78      North America
    France      211     55      Europe
    Japan       144     120     Asia
    Germany     96      61      Europe
    England     94      56      Europe
\end{file}

在这一章的剩下部分, 如果没有显式给出输入数据, 默认将 \filename{countries}
作为输入数据.

\subsection{程序格式}
\label{subsec:program_format}

\patact 语句, 以及动作内的语句通常用换行符分隔, 但是若干条语句也可出现在同
一行, 只要它们之间用分号分开即可. 一个分号可以放在任何语句的末尾.

动作的左花括号必须与它的模式在同一行; 而剩下的部分, 包括右花括号, 则可以出
现在下面几行.

空行会被忽略; 它们可在插入在语句之前或之后, 用来提高程序的可读性. 空格与制
表符可以出现在操作符与操作数的周围, 同样也是为了提高可读性.

注释可以出现在任意一行的末尾. 一个注释以井号开始, 以换行符结束, 正如
\begin{awkcode}
    { print $1, $3 }    # print country name and population
\end{awkcode}

一条长语句可以分散成多行, 只要在断行处插入一个反斜杆即可:
\begin{awkcode}
    { print \
            $1,     # country name
            $2,     # area in thousands of square miles
            $3 }    # population in millions
\end{awkcode}
正如这个例子所呈现的那样, 语句可以在逗号之后断行, 并且注释可以插入在断行的
末尾.

在这本书里我们用到了若干种格式化风格, 之所以这样做, 一方面是为了比较不同的
风格之间的差异, 另一方面是为了避免程序占用过多的行. 对于比较短小的程序,
就像本章中出现过那些例子, 格式并不是非常重要, 但是一致性与可读性对于大程序
的管理非常有帮助.

\section{模式}
\label{sec:patterns}
\marginpar{23}

模式控制动作的执行. 这一小节描述模式的6种类型, 以及匹配它们的条件.
\begin{summary}{模式汇总}
    \begin{enumerate}
        \item \BEGIN \verb'{' \stmt \verb'}'
            在输入被读取之后, \stmt 被执行一次.
        \item \END \verb'{' \stmt \verb'}'
            当所以输入被读取完毕之后, \stmt 被执行一次.
        \item \expr \verb'{' \stmt \verb'}'
            每碰到一个使 \expr 为真的输入行, \stmt 就被执行. \expr 为真指
            的是其值非零或非空.
        \item \verb'/'\regexpr\verb'/' \verb'{'
            \stmt \verb'}'
            当碰到这样一个输入行时, \stmt 就被执行: 输入行含有一段字符串,
            而该字符串可以被 \regexpr 匹配.
        \item \textit{compound pattern} \verb'{' \stmt \verb'}'
            一个复合模式将表达式用 \AND (AND), \OR (OR), \NOT (NOT), 以及
            括号组合起来; 当 \textit{compound pattern} 为真时, \stmt 执行.
        \item \pat$_1$\verb',' \pat$_2$ \verb'{' \stmt \verb'}'
            一个范围模式匹配多个输入行, 这些输入行从匹配 \pat$_1$ 的行开始,
            到匹配 \pat$_2$ 的行结束, 包括这两行; 对这其中的每一行执行
            \stmt.
    \end{enumerate}
    \BEGIN 与 \END 不与其他模式组合. 一个范围模式不能是其他模式的一部分.
    \BEGIN 与 \END 是唯一两个不能省略动作的模式.
\end{summary}

\subsection{\BEGIN 与 \END}
\label{subsec:begin_and_end}

\BEGIN 与 \END 这两个模式匹配任何输入行. 实际情况是, 当 awk 从输入读取数据
之前, \BEGIN 里的语句被执行; 当所有输入数据被读取完毕, \END 里的语句被执行.
于是, \BEGIN 与 \END 分别提供了一种控制初始化与扫尾的方式. \BEGIN 与 \END
不能与其他模式作组合.
