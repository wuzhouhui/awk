\chapter{报表与数据库}
\label{reports_and_databases}

\marginpar{89}
这章展示如何使用 awk 从文件中提取信息, 并生成报表, 我们把重点放在表格数据,
但是同样的技术也可以用在更加复杂的输入格式上. 本章的主题是开发一个可以与
其他程序配合使用的程序, 我们将会看到大量的, 工作中会经常遇到的数据处理问题,
这些问题很难一步解决, 但是如果多次遍历数据, 就相对比较容易一些.

本章的第一部分讨论如何扫描单个文件来生成报表, 虽然报表的最终格式的确需要花点
心思, 但是其实扫描步骤也是挺复杂的. 第二部分讨论如何从多个相关的文件中
收集数据, 我们考虑用一种比较一般的方法来解决这个问题, 通过把文件组当成
关系数据库, 这样做的好外是字段可以用名字来标识, 而不是数字.

\section{报表生成}
\label{sec:generating_reports}

Awk 可以从文件中挑选数据, 并将挑选到的数据格式化成报表. 我们将使用一个
三步骤过程来生成报表: 准备, 排序, 格式化. 准备步骤包括选择数据, 可能的话
还会对数据进行一些运算, 进而得到期望的信息; 如果我们想让数据按照某种特定
顺序排列, 就必须使用排序步骤, 排序操作可以通过将准备阶段的输出输送给系统
的排序命令来完成; 格式化操作由第 2 个 awk 程序来完成, 它根据已排序的数据
生成报表. 为了详细说明, 在这一节 我们利用第 \ref{chap:the_awk_language} 
章的文件\filename{countries} 来生成几张报表.

\subsection{一个简单的报表}
\label{subsec:a_simple_report}

假设我们想要一张报表, 这张表包含了每个国家的人口, 面积, 及人口密度. 
我们还希望国家按照所在的大洲进行分组, 大洲按照字母顺序排列, 在每个大洲里,
国家按照人口密度的降序排列, 就像这样:
\marginpar{90}
\begin{shell}
    CONTINENT       COUNTRY    POPULATION    AREA    POP. DEN.
    Asia            Japan          120        144      833.3
    Asia            India          746       1267      588.8
    Asia            China         1032       3705      278.5
    Asia            USSR           275       8649       31.8
    Europe          Germany         61         96      635.4
    Europe          England         56         94      595.7
    Europe          France          55        211      260.7
    North America   Mexico          78        762      102.4
    North America   USA            237       3615       65.6
    North America   Canada          25       3852        6.5
    South America   Brazil         134       3286       40.8
\end{shell}

生成报表的前两个阶段由程序 \verb'prep1' 完成, 当文件 \filename{countries}
作为输出时, \verb'prep1' 检查相关的信息, 并对其进行排序:
\begin{awkcode}
    # prep1 - prepare countries by continent and pop. den.

    BEGIN { FS = "\t" }
          { printf("%s:%s:%d:%d:%.1f\n",
                $4, $1, $3, $2, 1000*$3/$2) | "sort -t: +0 -1 +4rn"
          }
\end{awkcode}
输出是一系列的行, 每行都包括 5 个字段, 用冒号分隔, 从左至右, 字段依次表示
大洲, 国家, 人口, 面积, 以及人口密度:
\begin{shell}
    Asia:Japan:120:144:833.3
    Asia:India:746:1267:588.8
    Asia:China:1032:3705:278.5
    Asia:USSR:275:8649:31.8
    Europe:Germany:61:96:635.4
    Europe:England:56:94:595.7
    Europe:France:55:211:260.7
    North America:Mexico:78:762:102.4
    North America:USA:237:3615:65.6
    North America:Canada:25:3852:6.5
    South America:Brazil:134:3286:40.8
\end{shell}
程序 \verb'prep1' 把输出直接输送给 \verb'sort' 命令, 参数 \verb'-t' 告诉 
\verb'sort' 把冒号作为字段分隔符, \verb'+0 -1' 表示把第1个字段作为排序的
主键, 参数 \verb'+4rn' 表示把第 5 个字段作为次要主键, 按照数值的逆序进行
排序. (在 \ref{sec:a_sort_generator} 节, 我们将展示一个生成排序的程序,
这个程序可以从单词描述中生成排序所需的参数列表)

如果你的系统不支持管道, 那就把 \verb'sort' 命令删除, 使用 \verb'print >'
\textit{file} 直接输出到文件中, 然后再利用单独的步骤对文件排序,
这个方法适用于本章的所有例子.

现在我们已经完成了三个步骤中的前两个: 准备与排序, 现在所要做的是把数据
格式化成我们想要的报表格式, 程序 \verb'form1' 做的正是这个工作:
\marginpar{91}
\begin{awkcode}
    # form1 - format countries data by continent, pop. den.

    BEGIN { FS = ":"
            printf("%-15s %-10s %10s %7s %12s\n",
                "CONTINENT", "COUNTRY", "POPULATION",
                "AREA", "POP. DEN.")
          }
          { printf("%-15s %-10s %7d %10d %10.1f\n",
                $1, $2, $3, $4, $5)
          }
\end{awkcode}
