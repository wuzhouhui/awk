% vim: ts=4 sts=4 sw=4 et tw=75

\chapter{快速入门}
\label{chap:an_awk_tutorial}

\marginpar{1}
Awk 是一种使用方便且表现力很强的编程语言, 它可以应用在许多种类的计算与数据
处理任务中. 这一章是一个简短的教程, 旨在让你尽可能快地写出自己的 awk 程序.
第 \ref{chap:the_awk_language} 章对整个 awk 语言进行描述, 剩下的章节展现了
在多种不同的领域中, 如何使用 awk 解决问题. 在这本书中出现的例子, 读者应该
会感到非常有用, 有趣且具有指导作用.

\section{开始}
\label{sec:getting_started}

有用的 awk 程序通常都很短, 只有一两行. 假设你有一个文件, 叫作
\filename{emp.data}, 这个文件包含有名字, 每小时工资 (以美元为单位),
工作时长, 每一行代表一个雇员的记录, 就像这样
\begin{file}
    Beth    4.00    0
    Dan     3.75    0
    Kathy   4.00    10
    Mark    5.00    20
    Mary    5.50    22
    Susie   4.25    18
\end{file}
现在你想要打印每位雇员的名字以及他们的报酬 (每小时工资乘以工作时长), 而雇员
的工作时长必须大于零. 这种类型的工作是 awk 的设计目标之一, 所以会很简单.
只要键入下面这一行即可:
\begin{awkcode}
    awk '$3 > 0 { print $1, $2 * $3 }' emp.data
\end{awkcode}
这一行命令告诉系统运行 awk 命令, 被运行的程序被单引号包围, 从输入文件
\marginpar{2}
\filename{emp.data} 获取数据. 被单引号包围的部分是一个完整的 awk 程序. 它由
一个单独的 \cterm{模式-动作 语句)} (\term{pattern-action statement}) 组成.
模式 \verb'$3 > 0' 匹配每一行输入, 如果该行的第三列, 或者说 \cterm{字段}
(\term{field}) 大于零, 则动作
\begin{awkcode}
    { print $1, $2 * $3 }
\end{awkcode}
就会为每一个匹配行打印第一个字段, 以及第二与第三个字段的乘积.

如果你想要打印没有工作的雇员的名字, 键入
\begin{awkcode}
    awk '$3 == 0 { print $1 }' emp.data
\end{awkcode}
模式 \verb'$3 == 0' 匹配第三个字段为零的行, 动作
\begin{awkcode}
    { print $1 }
\end{awkcode}
打印该行的第一个字段.

当你阅读这本书时, 请尝试运行并修改书中的程序. 由于大多数程序都很简短, 你可
以快速理解 awk 是怎么工作的. 在一个 Unix 系统终端上, 上面提到的两个例子看
起来就像这样
\begin{shell}
    $ awk '$3 > 0 { print $1, $2 * $3 }' emp.data
    Kathy 40
    Mark 100
    Mary 121
    Susie 76.5
    $ awk '$3 == 0 { print $1 }' emp.data
    Beth
    Dan
    $
\end{shell}
一行开头的字符 \verb'$' 是系统的提示符, 在你的机器上或许会不一样.

\subsection{AWK 程序的结构}
\label{subsec:the_structure_of_an_awk_program}

现在让我们回退一步, 看一下到底发生了什么. 在上面的命令行中, 被单引号包围的
部分是使用 awk 语言编写的程序. 这一章中的每一个 awk 程序都是由一个或多个
模式-动作语句组成的序列:
\begin{pattern}
    \textit{pattern} \texttt{\{} \textit{action} \texttt{\}}
    \textit{pattern} \texttt{\{} \textit{action} \texttt{\}}
    ...
\end{pattern}
awk 的基本操作是在由输入行组成的序列中, 一个接一个地扫描每一行, 搜索被程序
中的模式 \cterm{匹配} (\term{match}) 的行.
\cterm{匹配}的精确含义依赖于问题中 的模式, 对于模式, 例如 \verb'$3 > 0',
意味着条件为真.

\marginpar{3}
每一个输入行轮流被每一个模式测试. 每匹配一个模式, 对应的动作 (可能包含多个
步骤) 就会执行. 然后下一行被读取, 匹配重新开始. 这个过程会一起持续
到所有的输入被读完为止.

上面的程序是模式与动作的典型例子
\begin{awkcode}
    $3 == 0 { print $1 }
\end{awkcode}
是一个单一的模式-动作语句, 对于每一个第3个字段为零的行, 它的第一个字段都会
被打印.

在一个模式-动作语句, 模式或动作可以省略其一, 但不能两者同时被省略. 如果一个
模式没有动作, 例如
\begin{awkcode}
    $3 == 0
\end{awkcode}
会将每一个匹配行 (也就是条件判断为真的行) 打印出来. 这个程序会被文件
\filename{emp.data} 中第3个字段为0的两行打印出来:
\begin{file}
    Beth    4.00    0
    Dan     3.75    0
\end{file}

如果只有动作而没有模式, 例如
\begin{awkcode}
    { print $1 }
\end{awkcode}
对于每一个输入行, 动作 (在这个例子里是打印第1个字段) 都会被执行.

因为模式与动作都是可选的, 所以用花括号将动作包围起来, 以便区分两者.

\subsection{运行 AWK 程序}
\label{running_an_awk_program}

运行一个 awk 程序有多种方式. 你可以键入下面这种形式的命令
\begin{pattern}
    \texttt{awk} \texttt{'}\textit{program}\texttt{'} \textit{input files}
\end{pattern}
对指定的输入文件的每一行运行程序 \textit{program}. 例如你可以键入
\begin{awkcode}
    awk '$3 == 0 { print $1 }' file1 file2
\end{awkcode}
来打印文件 \filename{file1} 与 \filename{file2} 的每一行的第一个字段,
如果该行的第3个字段为0的话.

你也可以在命令行上省略输入文件, 只要键入
\begin{pattern}
    \texttt{awk} \texttt{'}\textit{program}\texttt{'}
\end{pattern}
在这种情况下, awk 会将 \textit{program} 应用到你接下来在终端输入的内容上面,
直到你键入一个文件结束标志 (Unix 系统是组合键 control-d). 下面是一个在 Unix
的例子
\marginpar{4}
\begin{shell}
    $ awk '$3 == 0 { print $1 print $1 }'
    Beth    4.00    0
    Beth
    Dan     3.75    0
    Dan
    Kathy   3.75    10
    Kathy   3.75    0
    Kathy
    ...
\end{shell}
% TODO: 加粗显示未实现
被计算机打印的字符加粗显示.

这种行为使得用 awk 试验非常容易: 键入程序与数据, 看一下会发生什么. 我们再
次建议你尝试运行并修改书中的程序.

注意到, 命令行中的程序被单引号包围. 这种行为可以保护程序中的字符 (例如
\verb'$') 被 shell 解释, 也可以让程序的长度多于一行.

当程序的长度比较短时 (只有几行), 这种安排会比较方便. 如果程序比较长, 更好的
做法是将它们放在一个单独的文件中, 例如文件名可以叫作 \filename{progfile},
运行时只要键入
\begin{pattern}
    \texttt{awk -f progfile} \textit{optional list of files}
\end{pattern}
选项 \verb'-f' 告诉 awk 从文件中提取程序. 在 \filename{progfile} 出现的地方
可以是任意的文件名.

\subsection{错误}
\label{subsec:errors}

如果你在 awk 程序犯了一个错误, awk 会显示一个诊断信息. 例如, 如果你打错了一
个花括号, 就像这样
\begin{awkcode}
    awk '$3 == 0 [ print $1 }' emp.data
\end{awkcode}
你将会收到一个这样的消息
\begin{file}
    awk: syntax error at source line 1
    context is
            $3 == 0 >>> [ <<<
            extra }
            missing ]
    awk: Bailing out at source line 1
\end{file}
"Syntax error" 意味着你犯了一个语法错误, 这个错误被发现的地方用
\verb'>>> <<<' 标记. "Bailing out" 意味着无法恢复. 有时候, 你可以得到更多的
关于错误的信息, 例如报告了一个不匹配的花括号或括号.

由于发生了语法错误, awk 不会尝试执行这个程序.
然而有些错误直到运行时才会检测到. 例如, 你的程序尝试用 0 作除数, 这时候 awk
会停止处理, 接着报告输入行的行号, 以及程序中尝试进行除法运算的代码所在的
行号.

\section{简单的输出}
\label{sec:simple_output}

\marginpar{5}
本章的余下部分包含了一系列简短并且典型的 awk 程序, 这些程序都是基于对
\filename{emp.data} 文件的处理. 我们会简单地介绍这些程序是怎么工作的, 但
这些例子主要用于阐述有用的操作, 这些操作很容易用 awk 完成, 包括打印字段,
选择输入, 以及变换数据. 我们不会展现 awk 所能做的所有事情, 也不会对这里所
表现出来的特定的东西作过多的细节探究. 但是当你阅读完这一章时, 你将有能力
完成相当数量的工作, 而且会发现阅读后面的章节变得更加容易.

我们主要将程序显示出来, 而不是完整的命令行. 在每一种情况下, 程序或者可以
被包围在一对单引号中, 作为 \awk 命令的第一个参数来运行, 也可以将其放入一
个文件中, 通过带有 \verb'-f' 选项的 \awk 命令来运行.

在 awk 中只有两种类型的数据: 数值与由字符组成的字符串. 文件
\filename{emp.data} 是很典型的待处理数据, 它既含有单词, 也包括数值, 且字
段之间通过制表符或空格分隔.

awk 从它的输入中每次读取一行, 将行分解为一个个的字段, 默认将字段看作是不含
空白或制表符的字符组成的序列. 当前输入行的第一个字段叫作 \verb'$1', 第二个
是 \verb'$2', 依次类推. 一整行记为 \verb'$0'. 每行的字段数有可能不一样.

通常情况下, 我们需要做的是打印每一行的部分或全部字段, 也可能会做一些计算.
这一节中的所有程序都是这种形式.

\subsection{打印每一行}
\label{subsec:printing_every_line}

如果一个动作没有模式, 对于每一个输入行, 该动作都会被执行. 语句 \print 会
打印每一个当前输入行, 所以程序
\begin{awkcode}
    { print }
\end{awkcode}
会将它所有的输入打印到标准输出. 因为 \verb'$0' 表示一整行, 所以程序
\begin{awkcode}
    { print $0 }
\end{awkcode}
做的也是同一件事.

\subsection{打印某些字段}
\label{subsec:printing_certain_fields}

在一个 \print 语句中可以将多个项目打印到同一个输出行中. 打印每一个输入行
的第1与第3个字段的程序是
\begin{awkcode}
    { print $1, $3 }
\end{awkcode}
当 \filename{emp.data} 作为输入时, 它会输出
\marginpar{6}
\begin{file}
    Beth 0
    Dan 0
    Kathy 10
    Mark 20
    Mary 22
    Susie 18
\end{file}
在 \print 语句中由逗号分隔的表达式, 在输出时, 默认用一个空白符分隔. 由
\print 打印的每一行都由一个换行符终止. 这些默认行为都可以修改, 我们将在第
\ref{chap:the_awk_language} 章讨论如何修改.

\subsection{\texttt{NF}, 字段的数量}
\label{subsec:nf_the_number_fields}

有时候, 你必须总是通过 \verb'$1', \verb'$2' 这样的形式引用字段, 但是任何表
达式都可以出现在 \verb'$' 的后面, 来指明一个字段的号码; 表达式被求值, 求出
的值被当作字段的号码. Awk 计算当前输入行的字段数量, 并将它存储在一个内建的
变量中, 这个变量叫作 \nf. 因此程序
\begin{awkcode}
    { print NF, $1, $NF }
\end{awkcode}
将会打印每一个输入行的字段数量, 第一个字段, 以及最后一个字段.

\subsection{计算和打印}
\label{subsec:computing_and_printing}

你也可以用字段值进行计算, 并将计算得到的值放在输入语句中. 程序
\begin{awkcode}
    { print $1, $2 * $3 }
\end{awkcode}
是一个很典型的例如. 它会打印雇员的名字与报酬 (每小时工资乘以工作时长):
\begin{file}
    Beth 0
    Dan 0
    Kathy 40
    Mark 100
    Mary 121
    Susie 76.5
\end{file}
我们待会就会展示如果将输出做得更好看.

\subsection{打印行号}
\label{subsec:printing_line_numbers}

awk 提供了另一个内建变量, \nr, 这个变量计算到目前为止, 读取到的行的数量.
我们可以使用 \nr 和 \verb'$0' 为 \filename{emp.data} 的每一行加上行号:
\begin{awkcode}
    { print NR, $0 }
\end{awkcode}
输出就像这样:
\marginpar{7}
\begin{file}
    1 Beth    4.00    0
    2 Dan     3.75    0
    3 Kathy   4.00    10
    4 Mark    5.00    20
    5 Mary    5.50    22
    6 Susie   4.25    18
\end{file}

\subsection{将文本放入输出中}
\label{subsec:putting_text_in_the_output}

你也可以将单词放在字段与被计算的值的中间:
\begin{awkcode}
    { print "total pay for", $1, "is", $2 * $3 }
\end{awkcode}
输出
\begin{file}
    total pay for Beth is 0
    total pay for Dan is 0
    total pay for Kathy is 40
    total pay for Mark is 100
    total pay for Mary is 121
    total pay for Susie is 76.5
\end{file}
在 \print 语句中, 被双引号包围的文本会随着字段, 以及计算得到的值一起输出.

\section{更精美的输出}
\label{sec:fancier_output}

\print 用于简单快速的输出. 如果你想要格式化输出, 那么就需要使用 \printf
语句. 正如我们将要在 \ref{sec:output} 节看到的那样, \printf 几乎可以产生
任何种类的输出, 但在这一节, 我们仅仅展现它的一小部分能力.

\subsection{字段排队}
\label{subsec:lining_up_fields}

\printf 语句具有形式
\begin{pattern}
    \texttt{printf(}\textit{format}\texttt{,} \textit{value$_1$}\texttt{,}
    \textit{value$_2$}\texttt{, ... ,} \textit{value$_n$}\texttt{)}
\end{pattern}
\textit{format} 是一个字符串, 它包含按字面打印的文本, 中间散布着格式说明符,
格式说明符用于说明如何打印值. 一个格式说明符是一个 \verb'%', 后面跟着几个
字符, 这些字符控制一个 \textit{value} 的输出格式. 第一个格式说明符说明
\textit{value$_1$} 的输出格式, 第二个格式说明符说明 \textit{value$_2$} 的
输出格式, 依次类推. 于是, 格式说明符的数量应该和被打印的 \textit{value} 一
样多.

这个程序使用 \printf 打印每位雇员的报酬:
\begin{awkcode}
    { printf(" total pay for %s is $%.2f\n", $1, $2 * $3 }
\end{awkcode}
这个 \printf 语句的格式字符串包含两个格式说明符. 第一个格式说明符,
\marginpar{8}
\verb'%s', 是说将第一个值, \verb'$1', 以字符串的形式打印; 第二个格式说明符,
\verb'%.2f', 是说将第二个值, \verb'$2*$3', 以带有两位小数的数字打印. 格式
字符串的其他内容, 包括美元符, 按照字面值打印; 字符串末尾的 \verb'\n' 表示
换行符, 该符号使后来的输出从下一行开始. 当 \filename{emp.data} 作为输入时,
这个程序输出:
\begin{file}
    total pay for Beth is $0.00
    total pay for Dan is $0.00
    total pay for Kathy is $40.00
    total pay for Mark is $100.00
    total pay for Mary is $121.00
    total pay for Susie is $76.50
\end{file}
使用 \printf 不会自动产生空白符或换行符; 你必须自己创建它们. 不要忘了
\verb'\n'.

另外一个程序打印每位雇员的名字与报酬:
\begin{awkcode}
    { printf("%-8s $%6.2f\n", $1, $2 * $3) }
\end{awkcode}
第一个格式说明符, \verb'%-8s', 将名字左对齐输出, 占用8个字符的宽度. 第二个
格式说明符, \verb'%6.2f', 将报酬以带有两位小数的数字打印出来, 数字至少占用
6个字符的宽度:
\begin{awkcode}
    Beth     $  0.00
    Dan      $  0.00
    Kathy    $ 40.00
    Mark     $100.00
    Mary     $121.00
    Susie    $ 76.50
\end{awkcode}
更多的关于 \printf 的例子会慢慢加以介绍; 而关于 \printf 的完整描述在
\ref{sec:output} 节.

\subsection{输出排序}
\label{subsec:sorting_the_output}

设想一下你想要为每一位雇员打印所有的数据, 包括他的报酬, 报酬按照升序排列.
最简单的办法是使用 awk 在每一位雇员的记录前加上报酬, 然后再通过一个排序
程序进行排序, 在 Unix 中, 命令行
\begin{shell}
    awk '{ printf("%6.2f %s\n", $2 * $3, $0) }' emp.data | sort
\end{shell}
将 awk 的输出通过管道传递给 \texttt{sort} 命令, 最后输出:
\marginpar{9}
\begin{file}
      0.00 Beth    4.00    0
      0.00 Dan     3.75    0
    100.00 Mark    5.00    20
    121.00 Mary    5.50    22
     40.00 Kathy   4.00    10
     76.50 Susie   4.25    18
\end{file}

\section{选择}
\label{sec:selection}

Awk 的模式非常擅长从输入中选择感兴趣的行, 以进行进一步的处理. 因为一个没有
动作的模式会将所有匹配的行打印出来, 许多 awk 程序仅仅含有一个单独的模式.
这一节给出了一些具有实用价值的模式的例子.

\subsection{通过比较进行选择}
\label{subsec:selection_by_comparison}

这个程序使用一个比较模式来选择某些雇员的记录, 条件是他的每小时工资大于等于
\$5.00, 也就是记录的第二个字段的值大于等于 5:
\begin{awkcode}
    $2 >= 5
\end{awkcode}
它从 \filename{emp.data} 中选择这些行:
\begin{awkcode}
    Mark    5.00    20
    Mary    5.50    22
\end{awkcode}

\subsection{通过计算进行选择}
\label{subsec:selection_by_computation}

程序
\begin{awkcode}
    $2 * $3 > 50 { printf("$%.2f for %s\n", $2 * $3, $1) }
\end{awkcode}
打印那些报酬超过 \$50 的雇员:
\begin{file}
    $100.00 for Mark
    $121.00 for Mary
    $76.50 for Susie
\end{file}

\subsection{通过文本内容选择}
\label{subsec:selection_by_text_content}

除了数值选择, 你也可以选择那些包含特定单词或短语的输入行. 这个程序打印所有
第一个字段是 \texttt{Susie} 的行:
\begin{awkcode}
    $1 == "Susie"
\end{awkcode}
操作符 \texttt{==} 测试相等性. 你也可以寻找含有任意字母, 单词或短语的文本,
通过一个叫做\cterm{正则表达式} (\term{regular expressions}) 的模式来完成.
这个程序打印所有的, 在任意一个地方含有 \texttt{Susie} 的行:
\marginpar{10}
\begin{awkcode}
    /Susie/
\end{awkcode}
输出是
\begin{file}
    Susie   4.25    18
\end{file}
正则表达式可以用于指定精细得多的模式; \ref{sec:patterns} 包含一个完整的讨
论.

\subsection{模式的组合}
\label{subsec:combinations_of_patterns}

模式可以使用括号和逻辑运算符进行组合, 逻辑运算符包括 \AND, \OR, 和 \NOT,
分别表示 AND, OR, 和 NOT. 程序
\begin{awkcode}
    $2 >= 4 || $3 >= 20
\end{awkcode}
打印那些 \verb'$2' 至少为 4, 或者 \verb'$3' 至少为 20 的行:
\begin{awkcode}
    Beth    4.00    0
    Kathy   4.00    10
    Mark    5.00    20
    Mary    5.50    22
    Susie   4.25    18
\end{awkcode}
两个条件都满足的行只被打印一次. 将这个程序与下面这个程序作对比, 它包含两
个模式:
\begin{awkcode}
    $2 >= 4
    $3 >= 20
\end{awkcode}
如果某行对这两个条件都满足, 它会被打印两次
\begin{awkcode}
    Beth    4.00    0
    Kathy   4.00    10
    Mark    5.00    20
    Mark    5.00    20
    Mary    5.50    22
    Mary    5.50    22
    Susie   4.25    18
\end{awkcode}
注意程序
\begin{awkcode}
    !($2 < 4 && $3 < 20)
\end{awkcode}
打印的行不满足这样的条件: \verb'$2' 小于 4, 并且 \verb'$3' 也小于 20; 这个
条件判断与上面的第一个等价, 虽然在可读性方面差了一点.

\subsection{数据验证}
\label{subsec:data_validation}

真实的数据总是存在错误. 检查数据是否具有合理的值, 格式是否正确,
这种任务通常称作\cterm{数据验证} (\term{data validation}), 在这一方面 awk
是一个非常优秀的工具.

数据验证本质上是否定: 不打印带有期望的属性的行, 而是打印可疑行.  接下来的
程序使用比较模式, 将 5 条合理性测试应用到 \filename{emp.data} 的每一行:
\marginpar{11}
\begin{awkcode}
    NF != 3   { print $0, "number of fields is not equal to 3" }
    $2 < 3.35 { print $0, "rate is below minimum wage" }
    $2 > 10   { print $0, "rate exceeds $10 per hour" }
    $3 < 0    { print $0, "negative hours worked" }
    $3 > 60   { print $0, "too many hours worked" }
\end{awkcode}
如果数据没有错误, 就不会有输出.

\subsection{\BEGIN 与 \END}
\label{subsec:begin_and_end}

特殊的模式 \BEGIN 在第一个输入文件的第一行之前被匹配, \END 在最后一个输入
文件的最后一行被处理之后被匹配. 这个程序使用 \BEGIN 打印一个标题:
\begin{awkcode}
    BEGIN { print "NAME     RATE    HOURS"; print "" }
          { print }
\end{awkcode}
输出是
\begin{awkcode}
    NAME     RATE    HOURS

    Beth    4.00    0
    Dan     3.75    0
    Kathy   4.00    10
    Mark    5.00    20
    Mary    5.50    22
    Susie   4.25    18
\end{awkcode}
你可以在同一行放置多个语句, 语句之间用封号分开. 注意到, \verb'print ""' 打
印一个空行, 与一个单独的 \print 非常不同, 后者打印当前行.
